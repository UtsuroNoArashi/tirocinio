\begin{frame}{Algebra di Clifford}
    Sia fissato uno spazio vettoriale \(\mathbb{R}^{n}\), e sia 
    \(\{\mathbf{e}_{1}, \ldots, \mathbf{e}_{n}\}\) una sua base ortonormale.

    Da questi è possibile definire un nuovo spazio vettoriale,
    detto spazio di Clifford n-dimensionale (\(\Clifford\)).
    Nello specifico, \Clifford rappresenta l'insieme di tutti i 
    possibili sottospazi \(k\)-dimensionali, con \(k \le n\), 
    di tutte le possibili combinazioni dei vettori base.
\end{frame}
\begin{frame}{Multivettori: il caso bi- e tri-dimensionale}
    Si consideri lo spazio \(\mathbb{R}^{2}\) e sia \(\{\mathbf{e}_{1}, 
    \mathbf{e}_{2}\}\) una sua base ortonormale. 
    \'E noto che comunque presi \(\mathbf{a}, \mathbf{b} \in \mathbb{R}^{2}\),
    questi possano essere intesi come opportune combinazioni lineari dei vettori 
    base. Ossia
    \[
        \mathbf{a} = \alpha_{1}\mathbf{e}_{1} + \alpha_{2}\mathbf{e}_{2} 
        \qquad \mathbf{a} = \beta_{1}\mathbf{e}_{1} + \beta_{2}\mathbf{e}_{2}
    \]
    inoltre, sappiamo che 
    \begin{equation}\label{eq:1}
        \begin{aligned}
        \mathbf{ab} & = 
            (\alpha_{1}\mathbf{e}_{1} + \alpha_{2}\mathbf{e}_{2})
            (\beta_{1}\mathbf{e}_{1} + \beta_{2}\mathbf{e}_{2}) \\
            & = \alpha_{1}\beta_{1}\mathbf{e}_{1}\mathbf{e}_{1} +
                \alpha_{1}\beta_{2}\mathbf{e}_{1}\mathbf{e}_{2} +
                \alpha_{2}\beta_{1}\mathbf{e}_{1}\mathbf{e}_{2} + 
                \alpha_{2}\beta_{2}\mathbf{e}_{2}\mathbf{e}_{2}.
        \end{aligned}\
    \end{equation}
\end{frame}
\begin{frame}
    Defininedo i seguenti assiomi:
    \begin{enumerate}
        \item \(\mathbf{e}_{i}\mathbf{e}_{i} = 1\)
        \item \(\mathbf{e}_{j}\mathbf{e}_{i} = - \mathbf{e}_{i}\mathbf{e}_{j}\)
    \end{enumerate}
    \emph{Equazione \eqref{eq:1}} può essere riscritta come 
    \[
        \mathbf{ab} = (\alpha_{1}\beta_{1} + \alpha_{2}\beta_{2})
            + (\alpha_{1}\beta_{2} - \alpha_{2}\beta_{1})\mathbf{e}_{1}\mathbf{e}_{2}
    \]
\end{frame}
