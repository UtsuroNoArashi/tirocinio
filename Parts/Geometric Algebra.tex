\begin{frame}{Algebra di Clifford}
    Sia fissato uno spazio vettoriale \(\mathbb{R}^{n}\), e sia 
    \(\{\Vb{e}_{1}, \ldots, \Vb{e}_{n}\}\) una sua base ortonormale.

    Da questi è possibile definire un nuovo spazio vettoriale,
    detto spazio di Clifford n-dimensionale (\(\Clifford\)).
    Nello specifico, \Clifford rappresenta l'insieme di tutti i 
    possibili sottospazi \(k\)-dimensionali, con \(k \le n\), 
    di tutte le possibili combinazioni dei vettori base.
\end{frame}
\begin{frame}{Multivettori: il caso bi- e tri-dimensionale}
    Si consideri lo spazio \(\mathbb{R}^{2}\) e sia \(\{\Vb{e}_{1}, 
    \Vb{e}_{2}\}\) una sua base ortonormale. 
    \'E noto che comunque presi \(\Vb{a}, \Vb{b} \in \mathbb{R}^{2}\),
    questi possano essere intesi come opportune combinazioni lineari dei vettori 
    base. Ossia
    \[
        \Vb{a} = \alpha_{1}\Vb{e}_{1} + \alpha_{2}\Vb{e}_{2} 
        \qquad \Vb{a} = \beta_{1}\Vb{e}_{1} + \beta_{2}\Vb{e}_{2}
    \]
    inoltre, sappiamo che 
    \begin{equation}\label{eq:1}
        \begin{aligned}
        \Vb{ab} & = 
            (\alpha_{1}\Vb{e}_{1} + \alpha_{2}\Vb{e}_{2})
            (\beta_{1}\Vb{e}_{1} + \beta_{2}\Vb{e}_{2}) \\
            & = \alpha_{1}\beta_{1}\Vb{e}_{1}\Vb{e}_{1} +
                \alpha_{1}\beta_{2}\Vb{e}_{1}\Vb{e}_{2} +
                \alpha_{2}\beta_{1}\Vb{e}_{1}\Vb{e}_{2} + 
                \alpha_{2}\beta_{2}\Vb{e}_{2}\Vb{e}_{2}.
        \end{aligned}\
    \end{equation}
\end{frame}
\begin{frame}
    Defininedo i seguenti assiomi:
    \begin{enumerate}
        \item \(\Vb{e}_{i}\Vb{e}_{i} = 1\)
        \item \(\Vb{e}_{j}\Vb{e}_{i} = - \Vb{e}_{i}\Vb{e}_{j}\)
    \end{enumerate}
    \emph{Equazione \eqref{eq:1}} può essere riscritta come 
    \begin{equation}\label{eq:2}
        \Vb{ab} = (\alpha_{1}\beta_{1} + \alpha_{2}\beta_{2}) + 
            (\alpha_{1}\beta_{2} - \alpha_{2}\beta_{1})\Vb{e}_{1}\Vb{e}_{2}
    \end{equation}
\end{frame}
\begin{frame}
    Da \emph{Equazione \eqref{eq:2}} segue che,
    \(\forall \Vb{a}, \Vb{b} \in \mathbb{R}^{2}\)
    il loro prodotto risulti essere la somma di un termine scalare
    (\(\alpha_{1}\beta_{1} + \alpha_{2}\beta_{2}\)) e da un termine 
    \((\alpha_{1}\beta_{2} - \alpha_{2}\beta_{1})\Vb{e}_{1}\Vb{e}_{2}\).

    Dando un'interpretazione geometrica, 
    \((\alpha_{1}\beta_{2} - \alpha_{2}\beta_{1})\) descrive l'area 
    del rettandolo definita dai vettori \(\Vb{e}_{1}, \Vb{e}_{2}\);
    segue che (\(\Vb{e}_{1}, \Vb{e}_{2}\)) definiscono il piano
    su cui giace l'area.

    In conclusione, \(\Vb{e}_{1}\Vb{e}_{2}\) rappresenta un'area 
    orientata in \(\mathbb{R}^{2}\) ed è definita \emph{bivettore}.
\end{frame}
\begin{frame}
    Un ragionamento analogo può essere fatto per \(\mathbb{R}^{3}\).

    Sia \{\(\Vb{e}_{1}, \Vb{e}_{2}, \Vb{e}_{3}\)\} una base ortonormale
    di \(\mathbb{R}^{3}\). 
    Considerati \(\Vb{a}, \Vb{b} \in \mathbb{R}^{3}\),
    questi saranno della forma 
    \[
        \Vb{a} = \alpha_{1}\Vb{e}_{1} + \alpha_{2}\Vb{e}_{2} 
            + \alpha_{3}\Vb{e}_{3}  
            \qquad 
        \Vb{b} = \beta_{1}\Vb{e}_{1} + \beta_{2}\Vb{e}_{2}
            + \beta_{3}\Vb{e}_{3}
     \]
     Considerandone il prodotto, e applicando gli assiomi definti poco sopra,
     segue 
     \[\begin{aligned}
         \Vb{ab} & = (\alpha_{1}\beta_{1} + \alpha_{2}\beta_{2} + \alpha_{3}\beta_{3}) \\ 
            & + (\alpha_{1}\beta_{2} - \alpha_{2}\beta_{1})\Vb{e}_{1}\Vb{e}_{2} \\
            & + (\alpha_{1}\beta_{3} - \alpha_{3}\beta_{1})\Vb{e}_{1}\Vb{e}_{3} \\
            & + (\alpha_{2}\beta_{3} - \alpha_{3}\beta_{2})\Vb{e}_{2}\Vb{e}_{3} \\
     \end{aligned}\]
\end{frame}
